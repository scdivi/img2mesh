\documentclass{standalone}

\usepackage{amsmath}

\usepackage{tikz}
\usetikzlibrary{math}
\usetikzlibrary{spy}
\usetikzlibrary{calc}
\usepackage{newtxtext}       %
\usepackage{newtxmath}       % selects Times Roman as basic font

% define colors
% bright scheme
\definecolor{cbblue}{RGB}{0,119,187}
\definecolor{cbcyan}{RGB}{51,187,238}
\definecolor{cbteal}{RGB}{0,153,136}
\definecolor{cborange}{RGB}{238,119,51}
\definecolor{cbred}{RGB}{204,51,17}
\definecolor{cbmagenta}{RGB}{238,51,119}
\definecolor{cbgray}{RGB}{187,187,187}

% alternative bright scheme
\definecolor{cb2indigo}{RGB}{51,34,136}
\definecolor{cb2cyan}{RGB}{136,204,238}
\definecolor{cb2teal}{RGB}{68,170,153}
\definecolor{cb2green}{RGB}{17,119,51}
\definecolor{cb2olive}{RGB}{153,153,51}
\definecolor{cb2sand}{RGB}{221,204,119}
\definecolor{cb2rose}{RGB}{204,102,119}
\definecolor{cb2wine}{RGB}{136,34,84}
\definecolor{cb2purple}{RGB}{170,68,153}

% dark scheme
\definecolor{cbdarkblue}{RGB}{34,34,85}
\definecolor{cbdarkcyan}{RGB}{34,85,85}
\definecolor{cbdarkgreen}{RGB}{34,85,34}
\definecolor{cbdarkyellow}{RGB}{102,102,51}
\definecolor{cbdarkred}{RGB}{102,51,51}
\definecolor{cbdarkgray}{RGB}{85,85,85}

% light scheme
\definecolor{paleblue}{RGB}{187,204,238}
\definecolor{palecyan}{RGB}{204,238,255}
\definecolor{palegreen}{RGB}{204,221,170}
\definecolor{paleyellow}{RGB}{238,238,187}
\definecolor{palered}{RGB}{255,204,204}
\definecolor{palegray}{RGB}{221,221,221}


\begin{document}
\begin{tikzpicture}

\tikzmath{
	\wambient=4;
	\h=(1/8)*\wambient;
	}

  % axes
  \draw[->] (0,0) -- (1.1*\wambient,0) node[below] {$x_1$} coordinate(x axis);
  \draw[->] (0,0) -- (0,1.1*\wambient) node[left] {$x_2$} coordinate(y axis);
  
  % subcells (from data)
  \begin{scope}[scale=\wambient,line join=bevel,ultra thin,fill=black!25,draw=black!25]
    \input{data/carotid_domain.dat}
  \end{scope}

  \draw[thin] (0,0) rectangle (\wambient,\wambient);
    
  % boundary (from data)
  \begin{scope}[scale=\wambient, draw=cbblue]
    \input{data/carotid_boundary_32_53_5.dat}
  \end{scope}
  \begin{scope}[scale=\wambient, line cap=round, draw=cbblue]
    \input{data/carotid_top.dat}
  \end{scope}
  \begin{scope}[scale=\wambient, line cap=round, draw=cbblue]
    \input{data/carotid_bottom.dat}
  \end{scope}
    
  \node at (7.0*\h,0.5*\h) {$\Omega_{\rm scan}$};
  \node at (3.5*\h,1.5*\h) {$\Omega$};
  \node[cbblue] at (1.5*\h,2.5*\h) {$\partial \Omega$};

\end{tikzpicture}
\end{document}
